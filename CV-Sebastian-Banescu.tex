\documentclass[helvetica,english,nologo,notitle,totpages]{europecv2013}
\usepackage[T1]{fontenc}
\usepackage{graphicx}
\usepackage[a4paper,top=1.27cm,left=1cm,right=1cm,bottom=2cm]{geometry}
\usepackage[english]{babel}
\usepackage{bibentry}
\usepackage{natbib}

\renewcommand{\ttdefault}{phv} % Uses Helvetica instead of fixed width font

%\ecvname{Banescu, Sebastian\includegraphics{photo.jpg}}
\ecvname{Sebastian Banescu}
%\ecvfootername{Sebastian Banescu \quad Date:~16.12.2014}
\ecvaddress{Hochbr\"{u}cker Weg 2, 85396 Eching, Germany}
\ecvtelephone{(+49) 176 4152 5239}
\ecvemail{banescusebi@gmail.com}
%\ecvhomepage{\href{https://www22.in.tum.de/banescu}{www22.in.tum.de/banescu}}
\ecvlinkedin{\href{https://de.linkedin.com/in/sebastianbanescu}{de.linkedin.com/in/sebastianbanescu}}
%\ecvdateofbirth{20 October 1987}
%\ecvgender{Male}
\ecvpicture[width=1.5cm]{photo.jpg}
%\ecvfootnote{For more information go to \url{http://europass.cedefop.eu.int}\\
%\textcopyright~European Communities, 2003.}

\begin{document}
\selectlanguage{english}

\begin{europecv}
\ecvpersonalinfo[10pt]
%\ecvitem{\large\textbf{Desired employment/ Occupational~field}}{\large\textbf{}}

\ecvsection{Education and Training}
\ecvitem{Dates}{October 2014 - April 2017}
\ecvitem{Title of qualification awarded}{\textbf{Dr.~rer.~nat.} \textit{``summa cum laude''}}
\ecvitem{Name of organization}{\textbf{Technical University of Munich, Germany}, Center for Doctoral Studies in Informatics and its Applications (CeDoSIA) Graduate School, Faculty of Informatics}
\ecvitem{PhD Thesis Title}{Characterizing the Strength of Software Obfuscation Against Automated Attacks}
\ecvitem{}{}
\ecvitem{Dates}{September 2010 - August 2012}
\ecvitem{Title of qualification awarded}{\textbf{MSc.Information Security Technologies} \textit{``cum laude''} (GPA: 8.5 of 10, Thesis: 9 of 10)}
%\ecvitem{Principal subjects/Occupational skills covered}{Theoretical and technical expertise in the area of digital communication in general and in information security technology in particular.}
\ecvitem{Scholarship}{Talent Scholarship Program, currently Amandus H. Lundqvist Scholarship Program}
\ecvitem{Name of organization}{\textbf{Technical University of Eindhoven, The Netherlands}, Faculty of Computer Science}
\ecvitem{MSc.~Thesis Title}{Decision Support for Privacy Auditing}
\ecvitem{}{}
\ecvitem{Dates}{October 2006 - July 2010}
\ecvitem{Title of qualification awarded}{\textbf{BSc.~Computer Science and Engineering} (GPA: 9.5 of 10, Thesis: 10 of 10)}
%\ecvitem{Principal subjects/Occupational skills covered}{The study and design of computer and network systems components from both hardware and software perspectives and multiple specialization alternatives such as: computer architecture design, software engineering, artificial intelligence, operating systems, database design, compiler design, transactional systems, computer networks and distributed systems.}
\ecvitem{Scholarship}{Merit-based and Performance-based scholarships due to academic results}
\ecvitem{Name of organization}{\textbf{Technical University of Cluj-Napoca, Romania}, Faculty of Computer Science}
\ecvitem{BSc.~Thesis Title}{Unpredictable Random Number Generator Applied in Hardware Resource Allocation}
\ecvitem{}{}

\ecvsection{Work Experience}
\ecvitem{Dates}{May 2017 - onward}
\ecvitem{Position}{\textbf{IT Security Specialist} - member of Connected Car Security Team}
\ecvitem{Employer}{\textbf{BMW AG, Germany} - Connected Drive Department}
\ecvitem{Responsibilities}{Developing IT security defenses against car hackers, for the BMW fleet.}
\ecvitem{}{}


\ecvitem{Dates}{April 2013 - April 2017}
\ecvitem{Position}{\textbf{Researcher / Teaching Assistant} - member of Software Engineering Chair}
\ecvitem{Employer}{\textbf{Technical University of Munich, Germany} - Faculty of Informatics}
\ecvitem{Responsibilities}{Collaborated with Google Chrome security team to develop solutions against browser hijacking malware.~Teaching assistance for MSc.~and BSc.~level courses. Co-developed ``Secure Coding'' lecture, which was awarded the TUM prize for teaching excellence.}
\ecvitem{}{}
\ecvitem{Dates}{September 2012 - March 2013}
\ecvitem{Position}{\textbf{Security Engineer} - member of Digital Video Broadcast team}
\ecvitem{Employer}{\textbf{TP Vision, The Netherlands} - Innovation Site Eindhoven}
\ecvitem{Responsibilities}{Secure design, integration and testing of key management, DRM, copy and content protection systems. Mainly used C/C++. Assessed compliance and robustness rules for new systems.}
\ecvitem{}{}
\ecvitem{Dates}{February 2012 - August 2012}
\ecvitem{Position}{\textbf{Master Thesis Intern} - member of the T-Clouds project team}
\ecvitem{Employer}{\textbf{Philips Research, The Netherlands} - Healthcare Information Management, Security Cluster}
\ecvitem{Responsibilities}{Developed secure logging and log aggregation module for the TClouds project co-financed
under EU FP7 and obtained \textbf{patent US20160134495} for it.
Developed a privacy infringement detection and quantification tool and published 2 peer-reviewed papers about it. Mainly used Java.}
\ecvitem{}{}
\ecvitem{Dates}{July 2011 - November 2011}
\ecvitem{Position}{\textbf{Intern Student} - member of Security \& Privacy team}
\ecvitem{Employer}{\textbf{Deloitte, The Netherlands} - Enterprise Risk Services}
\ecvitem{Responsibilities}{Manual and (semi-)automated penetration testing of web-applications. Developed a privacy escalation 
testing tool as a script for OWASP WebScarab. Developed a password brute-forcing script for iMacros FF and IE plug-in. Mainly used PHP.}
%\ecvitem{}{}
%\ecvitem{Dates}{2005 - 2008}
%\ecvitem{Occupation}{\textbf{Freelance Web Developer}}
%\ecvitem{Main activities}{Design and develop presentation and e-commerce websites. Technologies used: PHP, JavaScript, AJAX, HTML, CSS, MySQL, Joomla, Drupal, OpenCart, ZenCart, Magento, PrestaShop. \textbf{Rent-a-coder/V-worker account name: sk8er} (ranked higher than 96\% of peers).}

\ecvsection{Selected Projects}
\ecvitem{2017-onward}{Bilateral Project between BMW and TU Munich: \textit{Intrusion Detection for Connected Cars}}
\ecvitem{2015-2016}{Bilateral Project between Google Canada and TU Munich: \textit{Software Protection for Chrome Against Memory Tampering}}
\ecvitem{2014}{Bilateral Project between Siemens and TU Munich: \textit{Detecting Bugs in Native Software Using Symbolic Execution}}
\ecvitem{2013-2014}{Bilateral Project between Google Germany and TU Munich: \textit{Software Protection for Chrome Against Browser Hijacking Attacks}}
\ecvitem{2012}{EU FP7 Project: \textit{Trustworthy Clouds -- Privacy and Resilience for Internet-scale Critical Infrastructure} (TClouds) \url{http://cordis.europa.eu/project/rcn/97862_en.html}}
\ecvitem{2011-2012}{Dutch Government Project: \textit{Trusted HEalthCare Services} (COMMIT/THECS) \url{http://www.commit-nl.nl/projects/trusted-healthcare-services}}
\ecvitem{2008-2010}{Romanian Government Project: \textit{A High Performance System for Generation and Testing of Random Number Sequences for Cryptographic Applications} (CryptoRand) \url{http://cryptorand.utcluj.ro/}}

\ecvsection{Awards, Grants and Scholarships}
\ecvitem{2017}{\textbf{Jungwissenschaftler 2017} awarded by \textit{Stiftung Werner-von-Siemens-Ring}}
\ecvitem{2016}{\textbf{Outstanding paper award} at 32nd Annual Computer Security Applications Conference \newline (ACSAC)}
\ecvitem{2016}{\textbf{Best paper award} at 6th Software Security, Protection and Reverse Engineering Workshop (SSPREW)}
\ecvitem{2015}{\textbf{Google Grant} for funding a full-time PhD student for one year}
\ecvitem{2014}{\textbf{Siemens Grant} for funding a full-time PhD student for one semester}
\ecvitem{2014}{\textbf{Best Code Cracker of ISSISP 2014} award at the International Summer School on Information Security and Protection, Verona, Italy}
\ecvitem{2014}{\textbf{TU Munich Award for Excellence in Teaching}, awarded for newly developed ``Secure Coding'' lecture}
\ecvitem{2013}{\textbf{Google Grant} for funding a full-time PhD student for one year}
\ecvitem{2010-2012}{\textbf{Dutch Talent Scholarship Program}, currently Amandus H. Lundqvist Scholarship Program}
\ecvitem{2009}{\textbf{ERASMUS Scholarship} for summer internship at ENS Lyon}
\ecvitem{2007-2010}{Romanian government sponsored merit-based and performance-based scholarships due to outstanding academic results}

%\ecvitem{Level in national or international classification\footnote{If appropriate.}}{\ldots}
%\ecvitem{Dates}{2002 - 2006}
%\ecvitem{Title of qualification awarded}{High School Degree}
%\ecvitem{Principal subjects/Occupational skills covered}{The study of basic computer programming, database design and use of  Microsoft Office Suite and Microsoft Windows Operating System.}
%\ecvitem{Scholarship}{Merit-based due to high scores during entire period of studies}
%\ecvitem{Name and type of organization providing education and training}{Colegiul National "Mihai Eminescu", Satu Mare}
\ecvitem{}{}

\ecvsection{Peer-Reviewed Publications}
\ecvitem{Journals} {\textbf{Banescu, S}; de Dinechin, F; Pasca, B; Tudoran R; - \textit{Multipliers for Floating-Point Double Precision and Beyond on FPGAs}. ACM SIGARCH Computer Architecture News 38.4: 73-79, 2010}
\ecvitem{Conferences}{}
\ecvitem{1}{\textbf{Banescu, S}; Collberg, C; Pretschner, A; \textit{Predicting the Resilience of Obfuscated Code Against Symbolic Execution Attacks via Machine Learning}. In Proc.~of the USENIX Security Symposium (USENIX Sec), 2017}
\ecvitem{2}{\textbf{Banescu, S}; Ahmadvand, M; Pretschner, A; Shield, R; Hamilton, C; \textit{Detecting Patching of Executables without System Calls}. In Proc.~of the 7th ACM Conference on Data and Application Security and Privacy (CODASPY), 2017}
\ecvitem{3}{Ochoa, M; \textbf{Banescu, S}; Disenfeld, C; Barthe, G; Ganesh, V; \textit{Reasoning about Probabilistic Defense Mechanisms against Remote Attacks.} In Proc.~of  2nd IEEE European Symposium on Security and Privacy (EuroS\&P), 2017}
\ecvitem{4}{\textbf{Banescu, S}; Collberg, C; Ganesh, V; Newsham, Z; Pretschner, A; \textit{Code Obfuscation Against Symbolic Execution Attacks.} In Proc.~of 32nd Annual Computer Security Applications Conference (ACSAC), 2016 \textbf{\color{red} Outstanding Paper Award}}
\ecvitem{5}{\textbf{Banescu, S}; Wuechner, T; Salem, A; Guggenmos, M; Ochoa, M; Pretschner, A; \textit{A Framework for Empirical Evaluation of Malware Detection Resilience Against Behaviour Obfuscation}. In Proc.~of 10th International Conference on Malicious and Unwandted Software (MALWARE), 2015}
\ecvitem{6}{Fedler, R; \textbf{Banescu, S}; Pretschner, A; \textit{ISA2R: Improving Software Attack and Analysis Resilience via Compiler-Level Software Diversity}. In Proc.~of 34th International Conference on Safety, Reliability, and Security (SAFECOMP), 2015}
\ecvitem{7}{\textbf{Banescu, S}; Pretschner, A; Battre, D; Cazzulani, S; Shield, R; Thompson, G; \textit{Software-Based Protection against ``Changeware''}. In Proc.~of the 5th ACM Conference on Data and Application Security and Privacy (CODASPY), 2015}
\ecvitem{8}{\textbf{Banescu, S}; Ochoa, M; Kunze, N; Pretschner, A; \textit{Idea: Benchmarking indistinguishability obfuscation - A candidate implementation}. In Proc.~of the International Symposium on Engineering Secure Software and Systems (ESSoS), 2015}
\ecvitem{9}{\textbf{Banescu, S}; Petkovic, M; Zannone, N; \textit{Measuring Privacy Compliance Using Fitness Metrics}. Proc.~of the 10th International Conference on Business Process Management (BPM), 2012}
\ecvitem{10}{Suciu, A; \textbf{Banescu, S}; Marton, K; \textit{Unpredictable random number generator based on hardware performance counters}. Digital Information Processing and Communications (ICDIPC), 2011}
\ecvitem{11}{Tudoran, R; \textbf{Banescu, S}; Cret, O; Suciu, A; - \textit{Implementing True Random Number Generators by Overfilling the FPGA Chip}. Proc.~of the FPGA World 2009 International Conference (FPGA World), 2009}
\ecvitem{12}{Colesa, A; Tudoran, R; \textbf{Banescu, S} - \textit{Software Random Number Generation Based on Race Conditions}. Proc.~of the 10th International Symposium on Symbolic and Numeric Algorithms for Scientific Computing, September (SYNASC), 2008}
%\ecvitem{7}{Saplacan, G; Buzdugan, L.; Rusu, M.; Salomie, I.; Nedevschi, S.; Dinsoreanu, M.; Sebestyen, G.; Alban, C.; \textbf{Banescu, S.}; Daian, A.; Gasko, T.; Ghirisan, A.; Gorcea, I.; Tudoran, R.; Turian, V. - \textbf{Resource Configuration Tools for a Traceability System in Food Industry}. Proc.~of the Workshop on Traceability and Systems for Traceability, IEEE International Conference on Intelligent Computer Communication and Processing, August, 2008, Cluj-Napoca, Romania, pp 29, ISBN 978-973-133-358-82008}
\ecvitem{Workshops}{}
\ecvitem{1}{Salem, A; \textbf{Banescu, S}. \textit{Metadata Recovery From Obfuscated Programs Using Machine Learning}. In Proc.~of the 6th Software Security, Protection and Reverse Engineering Workshop (SSPREW@ACSAC), 2016 \textbf{\color{red}Best Paper Award}}
\ecvitem{2}{\textbf{Banescu, S}; Lucaci, C; Kr\"amer, B; Pretschner, A; \textit{VOT4CS: A Virtualization Obfuscation Tool for C\#}. In Proc.~of 2nd International Workshop on Software Protection (SPRO@CCS), 2016}
\ecvitem{3}{Ibrahim, A; \textbf{Banescu, S}; \textit{StIns4CS: A State Inspection Tool for C\#}. In Proc.~of 2nd International Workshop on Software Protection (SPRO@CCS), 2016}
\ecvitem{4}{Holling, D; \textbf{Banescu, S}; Probst, M; Petrovska, A; Pretschner, A; \textit{Nequivack: Assessing mutation score confidence}. In Proc.~of 9th International Conference on Software Testing, Verification and Validation Workshops (ICSTW), 2016}
\ecvitem{5}{Ganesh, V; \textbf{Banescu, S}; Ochoa, M; \textit{The Meaning of Attack Resistant Systems}.
In Proc.~of the 10th Workshop on Programming Languages Analysis for Security (PLAS@ECOOP), 2015}
\ecvitem{6}{\textbf{Banescu, S}; Ochoa, M; Pretschner, A; \textit{A Framework for Measuring Software Resilience Against Automated Attacks}. In Proc.~of the 1st International Workshop on Software Protection (SPRO@ICSE), 2015}
\ecvitem{7}{\textbf{Banescu, S}; Zannone, N; \textit{Measuring privacy compliance with process specifications}. Proc.~of the 7th International Workshop on Security Measurements and Metrics (MetriSec), 2011}


\ecvsection{Trainings Offered}
\ecvitem{2017}{\textbf{Invited trainer} at ``7th Software Security, Protection and Reverse Engineering Workshop'' (SSPREW) \url{http://www.ssprew.org/} collocated with ACSAC 2017, Orlando, Florida, USA}
\ecvitem{2016}{\textbf{Invited trainer} at ``Industrial Software Protection Workshop'' organized by Dolby Germany in collaboration with TU Munich, at Dolby office in Nuremberg, Germany}

\ecvsection{Invited Talks}
\ecvitem{Jul. 2017}{\textit{``Characterizing the Strength of Software Obfuscation Against Automated Attacks''} at Dagstuhl Seminar on ``Malware Analysis: From Large-Scale Data Triage to Targeted Attack Recognition'', Dagstuhl, Germany}
\ecvitem{Apr. 2017}{\textit{``Characterizing the strength of software obfuscation against symbolic execution attacks''} at Singapore University of Technology and Design (SUTD) by Dr.~Martin Ochoa, Singapore}
\ecvitem{Dec. 2016}{\textit{``Analysing (De-)Obfuscation via Machine Learning''} at Itestra GmbH Jour Fixe, Munich, Germany}
\ecvitem{Sep. 2016}{\textit{``Code Obfuscation Against Symbolic Execution Attacks''} at Friedrich-Alexander Universit\"{a}t (FAU) Erlangen by Prof.~Dr.-Ing.~Felix Freiling, Erlangen, Germany}

\ecvsection{Scientific Service}
\ecvitem{Program Committee}{PC Member of ``7th Software Security, Protection and Reverse Engineering Workshop'' (SSPREW) collocated with ACSAC 2017, Orlando, Florida, USA}
\ecvitem{External Reviewer}{
\begin{enumerate}
 \item MSCS '17: Journal of Mathematical Structures in Computer Science
 \item IFIPSEC '17: International Conference on ICT Systems Security and Privacy Protection
 \item DIST '16: Journal of Distributed Computing
 \item SACMAT '15, '17: ACM Symposium on Access Control Models and Technologies
 \item CloudCom '16: IEEE International Conference on Cloud Computing Technology and Science
 \item TDSC '13, '14, '15: Transactions on Dependable and Secure Computing
 \item CODASPY '14, '15: ACM Conference on Data and Application Security and Privacy
 \item NSS '14, '15:	The International Conference on Network and System Security
 \item QSIC '13: International Conference on Quality Software
 \item ESORICS '13: European Symposium on Security in Computer Security
 
\end{enumerate}
}
\ecvitem{Supervised Students}{
\begin{enumerate}
 \item Alexander Ungar (BSc.~thesis): \textit{Benchmarking Symbolic Execution Tools
on Custom Block Ciphers}, submitted on 15 May 2017
 \item Ilya Migal (MSc.~thesis): \textit{Prediction of automated deobfuscation \& tampering
time using machine learning}, submitted on 15 Mar 2017
 \item Carlo DiDomenico (MSc.~thesis): \textit{iOS Application Hardening via Obfuscation}, submitted on 15 Jan 2017
 \item Dennis Fischer (BSc.~thesis): \textit{Detecting Process Memory Tampering}, submitted on 15 Feb 2016
 \item Amjad Ibrahim (MSc.~thesis): \textit{Software Protection by Self-Checking}, submitted on 15 Dec 2015
 \item Aleieldin Salem (MSc.~thesis): \textit{Metadata Recovery of Transformations from Obfuscated Software via Machine Learning Techniques}, submitted on 21 Oct 2015
 \item Ren\`e Milzarek (Guided research): \textit{A Taxonomy of Browser Hijacking Malware}, submitted on 19 Oct 2015
 \item Ciprian Lucaci (MSc.~thesis): \textit{Software Protection by Virtualization Obfuscation}, submitted on 15 Oct 2015
 \item Marco Probst (BSc.~thesis): \textit{Checking Non-Equivalence of Software Programs using Symbolic Execution}, submitted on 12 Jun 2015
 \item Andreas Geiger (MSc.~thesis): \textit{Raising the Bar for Automated Attacks against Web Applications using Software Diversity}, submitted on 15 May 2015
 \item Marius Guggenmos (BSc.~thesis): \textit{Towards Testing Malware Detection Systems using Behavioral Obfuscation}, submitted on 15 Feb 2015
 \item Rafael Fedler (MSc.~thesis): \textit{Code Transformations and Software Diversity for Improving Software Attack and Analysis Resilience}, submitted on 15 Nov 2014 \textbf{\color{red} CAST-Förderpreis IT-Sicherheit 2015}
 \item Nils Kunze (BSc.~thesis): \textit{A Qualitative Study of Indistinguishability Obfuscation}, submitted on 15 Aug 2014
 \item Nils Vissman (MSc.~thesis): \textit{Software Integrity Protection using White-Box Cryptography}, submitted on 15 May 2014
\end{enumerate}
}

\ecvsection{Research Visits}
\ecvitem{Dates}{February - March 2016}
\ecvitem{Position}{\textbf{Visiting Research Scholar} worked with Prof.~Dr.~Saumya Debray and Prof.~Dr.~Christian Collberg on characterizing obfuscation strength via case-studies using binary executables.}
%\ecvitem{Principal subjects/Occupational skills covered}{Study and implement FPGA target abstractions for Altera Stratix II, III \& IV and Virtex 5. Devise, implement and test an arithmetic operator optimized for the Stratix architectures. All contributions are part of the FloPoCo Arithmetic Core Generator.}
\ecvitem{Name of organization}{\textbf{University of Arizona, Tucson, USA}, Faculty of Computer Science}
\ecvitem{}{}

\ecvitem{Dates}{September 2015}
\ecvitem{Position}{\textbf{Visiting Research Scholar} worked with Prof.~Dr.~Vijay Ganesh on employing symbolic execution and SAT/SMT solvers for the purpose of de-obfuscating binary executables.}
%\ecvitem{Principal subjects/Occupational skills covered}{Study and implement FPGA target abstractions for Altera Stratix II, III \& IV and Virtex 5. Devise, implement and test an arithmetic operator optimized for the Stratix architectures. All contributions are part of the FloPoCo Arithmetic Core Generator.}
\ecvitem{Name of organization}{\textbf{University of Waterloo, Canada}, Department of Electrical and Computer Engineering}
\ecvitem{}{}

\ecvitem{Dates}{June - September 2009}
\ecvitem{Position}{\textbf{ERASMUS Exchange student} worked with Prof. Dr. Florent de Dinechin. Developed a C++ tool to generate high precision multiplication operators (as VHDL code) for FPGAs.}
%\ecvitem{Principal subjects/Occupational skills covered}{Study and implement FPGA target abstractions for Altera Stratix II, III \& IV and Virtex 5. Devise, implement and test an arithmetic operator optimized for the Stratix architectures. All contributions are part of the FloPoCo Arithmetic Core Generator.}
%\ecvitem{Scholarship}{European Region Action Scheme for the Mobility of University Students (\small{ERASMUS})}
\ecvitem{Name of organization}{\textbf{Ecole Normale Superieure Lyon, France}, Laboratoire de l'Informatique du Parallélisme}

\ecvsection{Personal Skills and~Competences}
\ecvmothertongue[5pt]{Romanian}
\ecvlanguageheader{}
\ecvlanguage{English}{\footnotesize{Advanced (C2)}}{\footnotesize{Advanced (C2)}}{\footnotesize{Advanced (C1)}}{\footnotesize{Advanced (C1)}}{\footnotesize{Advanced (C1)}}
\ecvlanguage{German}{\footnotesize{Intermediate (C1)}}{\footnotesize{Intermediate (C1)}}{\footnotesize{Intermediate (B2)}}{\footnotesize{Intermediate (B2)}}{\footnotesize{Intermediate (B2)}}
%\ecvlanguage{Dutch}{\footnotesize{Beginner (A2)}}{\footnotesize{Beginner (A2)}}{\footnotesize{Beginner (A1)}}{\footnotesize{Beginner (A1)}}{\footnotesize{Beginner (A1)}}
%\ecvlanguagefooter[5pt]{(*)}
\ecvitem{}{}
\ecvitem{Programming and}{\emph{Intermediate: } Java, C, R, x86 Assembly, Bash Script, Python}
\ecvitem{Scripting Languages}{\emph{Beginner: } C\#, VHDL, Matlab, Prolog, Haskel, ML, Lisp \vspace{0.2cm}}
%\ecvitem{Web Technologies}{\emph{Intermediate:} PHP, HTML, JavaScript, CSS, AJAX, XML
%\vspace{0.2cm}}
\ecvitem{Black-Box Testing Tools}{\emph{Beginner:} Nessus, Burpsuite, ZAP, Wireshark, Sqlmap, Zenmap
\vspace{0.2cm}}
\ecvitem{White-Box Testing Tools}{\emph{Intermediate:} KLEE, S2E} \ecvitem{}{\emph{Beginner:} Fortify, RIPS, FindBugs
\vspace{0.2cm}}
\ecvitem{Reverse Engineering}{\emph{Intermediate:} IDA Pro, GDB, angr, Triton, JavaDecompiler
\vspace{0.2cm}}
%\ecvitem{Integrated Development}{\emph{Intermediate:} Eclipse, MS Visual Studio}
%\ecvitem{Environments}{\emph{Beginner:} vim, Matlab, Xilinx ISE
%\vspace{0.2cm}}
%\ecvitem{SCM Tools}{\emph{Intermediate:} Subversion (SVN), Git
%\vspace{0.2cm}}
%\ecvitem{DBMS\vspace{0.3cm}}{\emph{Intermediate:} MySQL, MS SQL Server
%\vspace{0.2cm}}
%\ecvitem{Operating Systems}{Linux Debian/Ubuntu, Windows, Android
%\vspace{0.2cm}}
%\ecvitem{Document Editing Software}{LaTeX editors WinEdt and Kile; Microsoft Office Suite; Open/Libre Office Suite
%\vspace{0.2cm}}

%\ecvitem{Soft Skills}{Good Communication Skills, Team Player, Detail Oriented, Public Speaking \& Presenting}

\ecvsection{Miscellaneous}
\ecvitem{Poster Presentations}{NOTE: The following posters are not accompanied by proceedings
\begin{enumerate} 
 \item Banescu S. \textit{Raising the Bar for Browser Hijacking}, Google PhD Student Summit on Web Application Security, Google Office, Munich Germany, April 2016
 \item Banescu S. \textit{Diverse Software Obfuscation: Attacks and Defenses}, 34th TUM Graduate School Kick-Off Seminar, Frauenchiemsee, Germany, February 2015
 \item Banescu S. \textit{Attacks on Software Obfuscation and Diversity}, 5th International Summer School on Information Security and Protection, Verona, Italy, July 2014
\end{enumerate}
}
\ecvitem{Middle-/High-School}{Participated in various \textbf{mathematics and informatics olympiads and contests} at county and national levels. Obtained notable awards including \textbf{1st, 2nd and 3rd prizes}
\vspace{0.1cm}}
\ecvitem{Volunteer Work}{Volunteer IT Consultant for League of Romanian Students Abroad (2010-2012)
\vspace{0.1cm}}
 %\item Microsoft .NET Summer Rally in 2008 organized by Microsoft Student Partners in Cluj-Napoca.
\ecvitem{}{Volunteer in civic cleaning campaigns in my home town
\vspace{0.2cm}}

\ecvitem{Recommendations}{Upon request from Prof.~Dr.~Alexander Pretschner, e-mail: alexander.pretschner@tum.de}
\ecvitem{}{Other 10 recommendations already available on Linkedin: de.linkedin.com/in/sebastianbanescu}
\ecvitem{Research Interests}{Software Protection, Reverse Engineering, Anomaly Detection
\vspace{0.2cm}}

%\ecvitem{\large Other interests}{Data Science, Psychology, Advertising}

%\ecvitem{\large Driver's license}{Category B, Since: 29 November 2005}

\end{europecv}


\end{document} 